\documentclass{article}
\usepackage[utf8]{inputenc}

\title{Røde Kors mast manual}
\author{Øystein Smith}
\date{September 2018}

\usepackage{natbib}
\usepackage{graphicx}
\usepackage{subcaption}
\captionsetup{compatibility=false}
\usepackage{cleveref}
\begin{document}

\maketitle

\section{Introduction}
This short manual is for the mast known as CMC TYPE 406/654/1 also known as AB-952. Originaly made for the AN/GRC-103(V) radio system. In \cref{fig:labels} the labels attached to the mast launcher (base) is shown.

\begin{figure}[h!]
	\centering
	\begin{subfigure}{.5\textwidth}
		\centering
		\includegraphics[width=\linewidth]{img/Label1.jpg}
		\caption{Hæren label}
		\label{fig:label1}
	\end{subfigure}%
    ~
	\begin{subfigure}{.5\textwidth}
		\centering
		\includegraphics[width=\linewidth]{img/Label2.jpg}
		\caption{Manufacturer label}
		\label{fig:label2}
	\end{subfigure}
	\caption{Labels on the mast}
	\label{fig:labels}
\end{figure}

\subsection{Components}
The mast consists of
\begin{itemize}
\item Launcher
\item Seven mast sections
\item Top section
\item Three short guys (marked with red)
\item Three long guys
\item Three nails
\item Three long plugs
\item Three short plugs
\item Hammer
\end{itemize}

\section{Preparing}
First find a suitable location to erect the mast. The location has to havea flat area for the launcher with some soil to attach the launcher to the ground with plugs. A couple of meters clearing around the launcher, and clearing for three guy wires spaced 120 degrees about eight meters out from the launcher.

Then position the launcher in the middle of the suitable location and remove the mast sections from the launcher.

Isert the three long nails (about 20cm long) in the three holes in the bottom of the launcher marked with yellow to attach the launcher to the ground and prevent it from sliding sideways.

Then take out the three short guys, marked with red. And attach (either end) to the rings marked with red on the launcher. Stretch the guys out to about 4-8 meters from the launcher spaced 120 degrees from eachother, preferably one straight into the wind direction.

Then attach the top section to the mast section that is in the launcher.
This is a good time to attach the antenna or other things that should be in the top of the mast. There is a good idea to attach a ring (or several)  with a rope at least twice the length of the mast so you can hoist up stuff in the mast. For example one end of a wire dipole antenna.

Then attach the three long guys to the (rings on the) top section of the mast.


\section{Extending the launcher}
To extend the launcer, first lift the mast section that is in the launcher up such that the bottom of the mast section is flush with the bottom guide ring. Lift the launcher up until it hits a block, about 50cm, and the end of the yellow line is visible, rotate the top of the launcher 90 degrees counter clockwise lift it up 10 more cm then rotate the top of the launcher 90 degrees clockwise such that the yellow lines line up again. Set it down and it should be supported on the ledge.


\section{Guying}
Drive the tree short plugs into the ground about 4-8 meters from the launcher spaced 120 degrees apart, with one guy into the wind direction. Attatch the guys to the plugs. Make sure the screw for tightening the guys is in the loosest position. Open the lock on the guy, and pull the guy wires tight. (The wires are getting old and stiff so this might require some strength). When all the guys have been hand tightened, use the screws to tighten until sufficently tight.

Drive down the long plugs about 8-10 meters from the launcher spaced about 120 degrees apart, with one guy into the wind direction. Attatch the guys to the plugs. Make sure there is enough slack in the guy wires to be able to jack the mast up to the full height.

\section{Jacking up the mast}
Make sure the arrows in the jacking mechanism both is pointing upwards.
Make sure the friction holders is angled downwards and is holding the full weight of the mast section in the laucher.
Extend the jack, align the connection points between the jack and the jacking mechanism and insert the pin.
Use the jacking handle to jack the mast up, when the bottom of the mast is aligned with the bottom of the jacking mechanism, insert a new mast section. Then continue jacking, and repeat.

\begin{figure}[h]
	\centering
	\includegraphics[width=\linewidth]{img/Pin.JPG}
	\caption{Pin, and lower arrow}
	\label{fig:pin}
\end{figure}

\begin{figure}[h]
	\centering
	\includegraphics[width=\linewidth]{img/Arrow2.JPG}
	\caption{Upper arrow}
	\label{fig:arrow2}
\end{figure}

\section{Lowering the mast}

\section{Stowing the mast}

\end{document}

